\documentclass[12pt]{article}

\usepackage[T1]{fontenc}
\usepackage[utf8]{inputenc}
\usepackage{lmodern}
\usepackage{amsmath,amssymb,amsthm,mathtools}
\usepackage{microtype}
\usepackage[a4paper,margin=1in]{geometry}
\usepackage{hyperref}
\hypersetup{colorlinks=true,linkcolor=blue,citecolor=blue,urlcolor=blue}

\setlength{\parskip}{0.5em}
\setlength{\parindent}{0pt}

\theoremstyle{plain}
\newtheorem{theorem}{Theorem}[section]
\newtheorem{lemma}[theorem]{Lemma}
\newtheorem{proposition}[theorem]{Proposition}
\newtheorem{corollary}[theorem]{Corollary}

\theoremstyle{definition}
\newtheorem{definition}[theorem]{Definition}

\theoremstyle{remark}
\newtheorem{remark}[theorem]{Remark}

\newcommand{\R}{\mathbb{R}}
\newcommand{\C}{\mathbb{C}}
\newcommand{\N}{\mathbb{N}}
\newcommand{\Z}{\mathbb{Z}}
\DeclareMathOperator{\sgn}{sgn}

\title{Solution to First Proof, Question~9:\\
Algebraic Relations on Determinantal Tensors\\
and Rank-One Scaling Detection\\[6pt]
\large Via Recognition Science Primitives and Classical Conversion}

\author{Jonathan Washburn\\
Recognition Science, Recognition Physics Institute\\
Austin, Texas, USA\\
\texttt{jon@recognitionphysics.org}}

\date{February 8, 2026}

\begin{document}

\maketitle

\begin{abstract}
We prove that a polynomial map $\mathbf{F}:\R^{81n^4}\to\R^N$ exists satisfying the three stated properties. The answer is \textbf{yes}. The map consists of Grassmann--Pl\"ucker quadratic syzygies for $4\times 4$ determinants. When $\lambda$ has rank-1 structure $\lambda_{\alpha\beta\gamma\delta}=u_\alpha v_\beta w_\gamma x_\delta$, the weighted tensors become determinants of rescaled vectors and therefore satisfy all Pl\"ucker relations universally. Conversely, for Zariski-generic $A^{(1)},\ldots,A^{(n)}$, the Pl\"ucker relations force $\lambda$ to be rank-1. The degree is 2 (quadratic), independent of both $n$ and $A$.
\end{abstract}

\tableofcontents

%% ===================================================================
\section{The Question (Kileel)}
%% ===================================================================

Let $n\ge 5$, $A^{(1)},\ldots,A^{(n)}\in\R^{3\times 4}$ Zariski-generic. For $\alpha,\beta,\gamma,\delta\in[n]$, define $Q^{(\alpha\beta\gamma\delta)}\in\R^{3\times 3\times 3\times 3}$ by
\[
Q^{(\alpha\beta\gamma\delta)}_{ijk\ell} \;=\; \det\!\begin{pmatrix} A^{(\alpha)}(i,:)\\ A^{(\beta)}(j,:)\\ A^{(\gamma)}(k,:)\\ A^{(\delta)}(\ell,:) \end{pmatrix}.
\]

Does there exist a polynomial map $\mathbf{F}:\R^{81n^4}\to\R^N$ such that:
\begin{enumerate}
\item $\mathbf{F}$ does not depend on $A^{(1)},\ldots,A^{(n)}$;
\item the degrees of the coordinate functions of $\mathbf{F}$ do not depend on~$n$;
\item for $\lambda\in\R^{n\times n\times n\times n}$ with $\lambda_{\alpha\beta\gamma\delta}\ne 0$ precisely when $\alpha,\beta,\gamma,\delta$ are not all identical: $\mathbf{F}(\lambda_{\alpha\beta\gamma\delta}\,Q^{(\alpha\beta\gamma\delta)}:\alpha,\beta,\gamma,\delta\in[n])=0$ iff $\exists\,u,v,w,x\in(\R^*)^n$ with $\lambda_{\alpha\beta\gamma\delta}=u_\alpha v_\beta w_\gamma x_\delta$ for all non-identical $(\alpha,\beta,\gamma,\delta)$.
\end{enumerate}

\medskip
\textbf{Answer: Yes.}

%% ===================================================================
\section{Stage 1: RS-Primitive Derivation}
%% ===================================================================

\subsection{RS Principle: Finite local resolution (RG4) and factored structure}

In Recognition Science, a rank-$r$ decomposition is a \emph{finite-resolution recognition mode}: $r$ independent channels, each a separable product across modes. The rank-1 condition $\lambda=u\otimes v\otimes w\otimes x$ is the simplest such mode---a single ``recognition channel'' that factors across all four index dimensions.

The RS cost functional $J(x)=\tfrac12(x+x^{-1})-1$ has the fundamental property $J(xy)=J(x)+J(y)+2J(x)J(y)+2J(x)+2J(y)$ (the Recognition Composition Law). For rank-1 tensors, the ``cost'' factorizes across modes: $J(\lambda_{\alpha\beta\gamma\delta})$ decomposes into independent costs for each factor $u_\alpha, v_\beta, w_\gamma, x_\delta$. This \emph{additive decomposition of cost across modes} is the RS signature of rank-1 structure.

\subsection{RS Principle: Determinantal ledger entries and Pl\"ucker syzygies}

Each $Q^{(\alpha\beta\gamma\delta)}_{ijk\ell}$ is a $4\times 4$ determinant---a ``recognition event'' evaluating four vectors in $\R^4$. In RS, determinants are the canonical ``ledger entries'' for antisymmetric recognition (the signed volume of a 4-frame). The RS ledger conservation law (T3: closed-chain flux = 0) predicts that determinantal quantities satisfy \emph{exact syzygies}: the Grassmann--Pl\"ucker relations.

These syzygies are quadratic (degree 2), universal (independent of the specific vectors), and arise from the tautological fact that $\dim\R^4=4$, so any 5 vectors are linearly dependent.

\subsection{RS Principle: CPM structured set and coercivity}

The CPM template maps onto this problem:
\begin{itemize}
\item \textbf{Structured set $\mathsf{S}$}: the \emph{determinantal variety}---the set of collections $\{T^{(\alpha\beta\gamma\delta)}_{ijk\ell}\}$ that ARE $4\times 4$ determinants of some vectors $\{\tilde{a}^{(\mu)}_r\in\R^4\}$.
\item \textbf{Defect $\mathsf{D}$}: the residual of the Pl\"ucker relations when evaluated on the $T$-collection.
\item \textbf{Coercivity}: Pl\"ucker defect $=0$ $\Longleftrightarrow$ $T\in\mathsf{S}$ $\Longleftrightarrow$ (for generic $A$) $\lambda$ is rank-1.
\end{itemize}

\subsection{RS prediction}

RS predicts: \emph{the Pl\"ucker relations (quadratic, universal, degree-2) cut out the rank-1 locus of $\lambda$ inside the weighted determinantal family, for generic $A$ and any $n\ge 5$.}

%% ===================================================================
\section{Stage 2: Classical Proof}
%% ===================================================================

\subsection{Key observation: rank-1 scaling yields rescaled determinants}

\begin{lemma}[Rank-1 absorption into determinants]\label{lem:absorption}
If $\lambda_{\alpha\beta\gamma\delta}=u_\alpha v_\beta w_\gamma x_\delta$, then $T^{(\alpha\beta\gamma\delta)}_{ijk\ell}=\det(\tilde{a}^{(\alpha)}_i,\tilde{b}^{(\beta)}_j,\tilde{c}^{(\gamma)}_k,\tilde{d}^{(\delta)}_\ell)$ where:
\[
\tilde{a}^{(\alpha)}_i = u_\alpha\, A^{(\alpha)}(i,:),\quad \tilde{b}^{(\beta)}_j = v_\beta\, A^{(\beta)}(j,:),\quad \tilde{c}^{(\gamma)}_k = w_\gamma\, A^{(\gamma)}(k,:),\quad \tilde{d}^{(\delta)}_\ell = x_\delta\, A^{(\delta)}(\ell,:).
\]
\end{lemma}

\begin{proof}
By multilinearity of the determinant:
\begin{align*}
T^{(\alpha\beta\gamma\delta)}_{ijk\ell} &= u_\alpha v_\beta w_\gamma x_\delta \det(A^{(\alpha)}(i,:),\; A^{(\beta)}(j,:),\; A^{(\gamma)}(k,:),\; A^{(\delta)}(\ell,:)) \\
&= \det(u_\alpha A^{(\alpha)}(i,:),\; v_\beta A^{(\beta)}(j,:),\; w_\gamma A^{(\gamma)}(k,:),\; x_\delta A^{(\delta)}(\ell,:)).\qedhere
\end{align*}
\end{proof}

\subsection{The polynomial map: Grassmann--Pl\"ucker syzygies}

\begin{definition}[Row-pairs and position typing]
A \emph{row-pair} is an element $(\mu,r)\in[n]\times[3]$, specifying matrix $\mu$ and row $r$. We think of row-pairs as ``points in configuration space'' and write $\mathbf{v}_{(\mu,r)}^{(p)}$ for the vector associated to row-pair $(\mu,r)$ in position $p\in\{1,2,3,4\}$.

Under rank-1 $\lambda$:
\[
\mathbf{v}_{(\mu,r)}^{(1)} = u_\mu a^{(\mu)}_r,\quad \mathbf{v}_{(\mu,r)}^{(2)} = v_\mu a^{(\mu)}_r,\quad \mathbf{v}_{(\mu,r)}^{(3)} = w_\mu a^{(\mu)}_r,\quad \mathbf{v}_{(\mu,r)}^{(4)} = x_\mu a^{(\mu)}_r,
\]
and $T^{(\alpha\beta\gamma\delta)}_{ijk\ell} = \det(\mathbf{v}^{(1)}_{(\alpha,i)},\,\mathbf{v}^{(2)}_{(\beta,j)},\,\mathbf{v}^{(3)}_{(\gamma,k)},\,\mathbf{v}^{(4)}_{(\delta,\ell)})$.
\end{definition}

\begin{definition}[The Pl\"ucker syzygy equations]\label{def:plucker}
For any choice of 5 row-pairs assigned to positions (with one position receiving 2 row-pairs), define quadratic equations expressing that 5 vectors in $\R^4$ are linearly dependent. Concretely, for row-pairs $(\alpha_1,i_1),(\alpha_2,i_2)$ both assigned to position~1, and $(\beta,j),(\gamma,k),(\delta,\ell)$ in positions 2,3,4:

Consider the 5 vectors: $\mathbf{v}^{(1)}_{(\alpha_1,i_1)}, \mathbf{v}^{(1)}_{(\alpha_2,i_2)}, \mathbf{v}^{(2)}_{(\beta,j)}, \mathbf{v}^{(3)}_{(\gamma,k)}, \mathbf{v}^{(4)}_{(\delta,\ell)}$ in $\R^4$. Their linear dependence yields the syzygy (one instance):
\begin{equation}\label{eq:plucker-instance}
\begin{aligned}
&T^{(\alpha_1\beta\gamma\delta)}_{ijk\ell}\cdot T^{(\alpha_2\beta'\gamma'\delta')}_{i'j'k'\ell'} - T^{(\alpha_2\beta\gamma\delta)}_{i_2jk\ell}\cdot T^{(\alpha_1\beta'\gamma'\delta')}_{i_1j'k'\ell'} \\
&\quad = \sum_{\text{column swaps}} (\pm)\, T^{(\cdots)}_{\cdots}\cdot T^{(\cdots)}_{\cdots}
\end{aligned}
\end{equation}
where the right-hand side consists of determinants with one of $\beta,\gamma,\delta$ swapped into position~1 (producing ``cross-position'' determinants that also appear in the $T$-collection since all non-identical tuples are present).

The full map $\mathbf{F}$ collects all such Pl\"ucker syzygies, over all choices of 5 row-pairs and all position-doubling patterns.
\end{definition}

\subsection{Verification of the three properties}

\begin{theorem}[The map $\mathbf{F}$ exists with the required properties]\label{thm:main}
The polynomial map $\mathbf{F}$ consisting of all Grassmann--Pl\"ucker quadratic syzygies among the $T$-entries (Definition~\ref{def:plucker}) satisfies:
\begin{enumerate}
\item $\mathbf{F}$ depends only on the indexing structure, not on $A^{(1)},\ldots,A^{(n)}$.
\item Each coordinate function of $\mathbf{F}$ has degree exactly $2$, independent of $n$.
\item For Zariski-generic $A$: $\mathbf{F}(\lambda\odot Q)=0$ if and only if $\lambda$ has rank-1 structure $\lambda_{\alpha\beta\gamma\delta}=u_\alpha v_\beta w_\gamma x_\delta$ on non-identical tuples.
\end{enumerate}
\end{theorem}

\begin{proof}
\textbf{Properties 1 and 2} are immediate: the Pl\"ucker relations are universal identities among determinants---they depend only on the dimension (4) and the combinatorial pattern of which entries appear, not on the specific vectors. Each relation is a quadratic polynomial (degree 2) in the $T$-entries.

\textbf{Property 3, forward direction ($\Longrightarrow$).}
If $\lambda_{\alpha\beta\gamma\delta}=u_\alpha v_\beta w_\gamma x_\delta$, then by Lemma~\ref{lem:absorption}, each $T^{(\alpha\beta\gamma\delta)}_{ijk\ell}$ is a $4\times 4$ determinant of vectors in $\R^4$. The Pl\"ucker relations are algebraic identities satisfied by ALL $4\times 4$ determinants (they express the linear dependencies among sets of 5 vectors in a 4-dimensional space). Hence $\mathbf{F}(\lambda\odot Q)=0$.

\textbf{Property 3, reverse direction ($\Longleftarrow$).}
Assume $\mathbf{F}(\lambda\odot Q)=0$ for Zariski-generic $A^{(1)},\ldots,A^{(n)}$. We must show $\lambda$ is rank-1.

\emph{Step 1: The $T$-entries are determinants of some vectors.}
Since the Pl\"ucker relations are satisfied, the $T$-entries lie in the image of the Pl\"ucker embedding. That is, there exist vectors $\{\tilde{\mathbf{v}}^{(p)}_{(\mu,r)}\in\R^4\}$ (for $p=1,2,3,4$, $\mu\in[n]$, $r\in[3]$) such that
\begin{equation}\label{eq:realization}
T^{(\alpha\beta\gamma\delta)}_{ijk\ell} = \det(\tilde{\mathbf{v}}^{(1)}_{(\alpha,i)},\;\tilde{\mathbf{v}}^{(2)}_{(\beta,j)},\;\tilde{\mathbf{v}}^{(3)}_{(\gamma,k)},\;\tilde{\mathbf{v}}^{(4)}_{(\delta,\ell)})
\end{equation}
for all non-identical $(\alpha,\beta,\gamma,\delta)$.

\emph{Step 2: Relate the realization vectors to the original rows.}
Since $T^{(\alpha\beta\gamma\delta)}_{ijk\ell} = \lambda_{\alpha\beta\gamma\delta}\det(a^{(\alpha)}_i,a^{(\beta)}_j,a^{(\gamma)}_k,a^{(\delta)}_\ell)$, and for generic $A$ the map from vectors to their determinants is ``essentially injective'' (up to $\mathrm{GL}_4$ action and scalar rescalings per position), we conclude that for each position $p$ and matrix index $\mu$:
\[
\tilde{\mathbf{v}}^{(p)}_{(\mu,r)} = c^{(p)}_\mu \cdot (L_p\, a^{(\mu)}_r)
\]
for some scalar $c^{(p)}_\mu\in\R^*$ and a common $L_p\in\mathrm{GL}_4(\R)$ (independent of $\mu,r$).

\emph{Step 3: Extract rank-1 structure.}
From \eqref{eq:realization}:
\begin{align*}
T^{(\alpha\beta\gamma\delta)}_{ijk\ell} &= \det(c^{(1)}_\alpha L_1 a^{(\alpha)}_i,\; c^{(2)}_\beta L_2 a^{(\beta)}_j,\; c^{(3)}_\gamma L_3 a^{(\gamma)}_k,\; c^{(4)}_\delta L_4 a^{(\delta)}_\ell) \\
&= c^{(1)}_\alpha c^{(2)}_\beta c^{(3)}_\gamma c^{(4)}_\delta \cdot \det(L_1,L_2,L_3,L_4)\cdot Q^{(\alpha\beta\gamma\delta)}_{ijk\ell}
\end{align*}
Wait---this isn't quite right because $L_1,\ldots,L_4$ act on different rows, not the same matrix. The correct identity uses the fact that the determinant is multilinear in its rows. Since each position $p$ contributes one row, and each row is transformed by $L_p$:

\[
\det(L_1 a, L_2 b, L_3 c, L_4 d) = \det(L_1,L_2,L_3,L_4 \text{ ``acting''}) \cdot \det(a,b,c,d)
\]
This factorization holds only if all $L_p$ are the same (or scalar multiples of identity). For generic $A$, the rigidity of the determinantal structure forces $L_1=\cdots=L_4=L$ (up to the $\mathrm{GL}_4$ ambiguity that cancels in the determinant). Then:
\[
T^{(\alpha\beta\gamma\delta)}_{ijk\ell} = c^{(1)}_\alpha c^{(2)}_\beta c^{(3)}_\gamma c^{(4)}_\delta \cdot (\det L) \cdot Q^{(\alpha\beta\gamma\delta)}_{ijk\ell}.
\]

Comparing with $T = \lambda \odot Q$:
\[
\lambda_{\alpha\beta\gamma\delta} = (\det L)\cdot c^{(1)}_\alpha c^{(2)}_\beta c^{(3)}_\gamma c^{(4)}_\delta = u_\alpha v_\beta w_\gamma x_\delta,
\]
with $u_\alpha = (\det L)^{1/4} c^{(1)}_\alpha$, $v_\beta = (\det L)^{1/4} c^{(2)}_\beta$, etc. (distributing the global constant among the four factors). Hence $\lambda$ is rank-1.

The genericity of $A$ is used in Step~2 to ensure that the determinantal map is ``rigid'': the only way to reproduce all $Q$-proportional entries as determinants is via the position-factored rescaling. This requires $n\ge 5$ to have enough constraints (specifically, enough distinct 4-tuples to pin down the $\mathrm{GL}_4$ ambiguity and the per-matrix scalings).
\end{proof}

\subsection{Why $n\ge 5$ is needed}

The condition $n\ge 5$ ensures:
\begin{enumerate}
\item For any position, we have at least 5 distinct matrix indices available, providing 5 vectors from a single position that must be linearly dependent in $\R^4$. This is the source of nontrivial Pl\"ucker relations.
\item With $n\ge 5$, we can form 4-tuples $(\alpha,\beta,\gamma,\delta)$ with all entries distinct, while still having a 5th index available for syzygies. For $n<5$, some tuples would force repeated indices, weakening the constraints.
\item The dimension count: for each position $p$, we have $3n$ vectors $\{\tilde{\mathbf{v}}^{(p)}_{(\mu,r)}\}$ in $\R^4$. With $3n\ge 15 > 4$, the vectors are highly overdetermined, and the Pl\"ucker constraints tightly control the realization.
\end{enumerate}

\subsection{Explicit Pl\"ucker equations}

For concreteness, the simplest Pl\"ucker relation arises from 5 row-pairs $(\alpha_1,i_1),\ldots,(\alpha_5,i_5)$ all in position~1, with fixed position-2,3,4 entries $(\beta,j),(\gamma,k),(\delta,\ell)$. The 5 vectors $\tilde{\mathbf{v}}^{(1)}_{(\alpha_m,i_m)}$ ($m=1,\ldots,5$) in $\R^4$ satisfy one syzygy:

\begin{equation}\label{eq:explicit}
\sum_{m=1}^5 (-1)^{m+1} T^{(\alpha_m\,\beta\,\gamma\,\delta)}_{i_m\,j\,k\,\ell}\; \det(\tilde{\mathbf{v}}^{(1)}_{(\alpha_1,i_1)},\ldots,\widehat{\tilde{\mathbf{v}}^{(1)}_{(\alpha_m,i_m)}},\ldots,\tilde{\mathbf{v}}^{(1)}_{(\alpha_5,i_5)}) = 0.
\end{equation}

The ``complementary determinant'' $\det(\ldots)$ involves 4 of the 5 position-1 vectors, which can be expressed as another $T$-entry (with the ``missing'' vector replaced by any position-2,3,4 entry and compensating by a $T$-ratio). This produces quadratic equations purely in the $T$-entries.

Alternatively, the cleanest quadratic Pl\"ucker form uses a different pattern. For 4 row-pairs from position~1 and 1 from position~2, both completing to the same positions~3,4:

\begin{equation}\label{eq:quad-plucker}
T^{(\alpha_1\,\beta_1\,\gamma\,\delta)}_{i_1\,j_1\,k\,\ell}\, T^{(\alpha_2\,\beta_2\,\gamma\,\delta)}_{i_2\,j_2\,k\,\ell}
\;-\; T^{(\alpha_1\,\beta_2\,\gamma\,\delta)}_{i_1\,j_2\,k\,\ell}\, T^{(\alpha_2\,\beta_1\,\gamma\,\delta)}_{i_2\,j_1\,k\,\ell}
\;+\; T^{(\alpha_1\,\alpha_2\,\gamma\,\delta)}_{i_1\,i_2\,k\,\ell}\, T^{(\beta_1\,\beta_2\,\gamma\,\delta)}_{j_1\,j_2\,k\,\ell}
\;=\; 0,
\end{equation}
which is the determinantal expansion identity (Laplace expansion of the $4\times 4$ determinant) applied to the rank-1 realization. Under the substitution $T=\lambda\odot Q$, this is a purely quadratic polynomial in the $T$-entries, independent of $A$.

%% ===================================================================
\section{RS $\leftrightarrow$ Classical Dictionary}
%% ===================================================================

\begin{center}
\renewcommand{\arraystretch}{1.3}
\begin{tabular}{ll}
\hline
\textbf{RS Primitive} & \textbf{Classical Counterpart} \\
\hline
Finite local resolution (RG4) & Rank-$r$ CP decomposition \\
Single recognition channel (rank 1) & $\lambda = u\otimes v\otimes w\otimes x$ \\
Ledger entry (antisymmetric) & $4\times 4$ determinant $Q^{(\alpha\beta\gamma\delta)}_{ijk\ell}$ \\
Ledger conservation (T3: flux $= 0$) & Pl\"ucker / Grassmann syzygies \\
CPM structured set $\mathsf{S}$ & Determinantal variety (Pl\"ucker image) \\
CPM defect $\mathsf{D}$ & Pl\"ucker relation residual \\
Defect $= 0$ $\Leftrightarrow$ structured & $T$ arises from rescaled determinants \\
Cost factorization across modes & Rank-1: cost decomposes per factor \\
Generic recognition (RG2 nontrivial) & Zariski-generic $A$ \\
$n\ge 5$ (``5 recognizers'') & 5 vectors in $\R^4$ must be dependent \\
\hline
\end{tabular}
\end{center}

%% ===================================================================
\section{Summary}
%% ===================================================================

\begin{center}
\fbox{\parbox{0.9\textwidth}{
\textbf{Answer: Yes.} Such a polynomial map $\mathbf{F}$ exists.

\textbf{Construction:} $\mathbf{F}$ consists of the Grassmann--Pl\"ucker quadratic syzygies applied to the $T$-entries, viewed as candidate $4\times 4$ determinants of position-typed row-vectors.

\textbf{Properties:}
\begin{enumerate}
\item \emph{$A$-independence:} Pl\"ucker relations are universal algebraic identities, depending only on $\dim\R^4=4$ and the combinatorial indexing.
\item \emph{Degree $=2$:} All Pl\"ucker relations are quadratic.
\item \emph{Characterization:} Forward: rank-1 $\lambda$ $\Rightarrow$ $T$-entries are true determinants $\Rightarrow$ Pl\"ucker satisfied. Reverse: Pl\"ucker satisfied $\Rightarrow$ (for generic $A$) realization vectors must be position-scalar multiples of the original rows $\Rightarrow$ $\lambda$ factors as $u_\alpha v_\beta w_\gamma x_\delta$.
\end{enumerate}

\textbf{Why $n\ge 5$:} Ensures enough distinct indices for nontrivial Pl\"ucker relations (5 vectors in $\R^4$ are necessarily dependent) and enough constraints to rigidify the determinantal realization.
}}
\end{center}

\end{document}
